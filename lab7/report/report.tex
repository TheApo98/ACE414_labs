\documentclass[11pt,a4paper]{report}

\usepackage{ucs}
\usepackage[utf8x]{inputenc}
\usepackage[greek,english]{babel}
\usepackage{eucal}
% \usepackage{tabularray}
\usepackage[table]{xcolor}
\usepackage{tabularray}
\usepackage{tabularx}


\newcommand{\gr}{\selectlanguage{greek}} % switch to greek
\newcommand{\en}{\selectlanguage{english}}  % switch to english
\newcommand{\notes}[1]{ \comment{ #1 } }
\newcommand{\latin}[1]{\en#1\gr}			% English text in greek text environment
\newcommand{\code}[1]{\texttt{\latin{#1}}}	% In-line code snippet
\newcommand{\unit}[1]{\text{\latin{ #1}}}	% For upright font in math mode

% set papper view
\usepackage{geometry}
\geometry{
	a4paper,
	left=20mm,
	right=20mm,
	top=15mm,
}

% set hyper link set up
\usepackage{hyperref}
\hypersetup{
    colorlinks,
    citecolor=black,
    filecolor=black,
    linkcolor=black,
    urlcolor=blue
}


\usepackage{csquotes} % \textquote{ } / \begin{displayquote} ... \end{displayquote}
\usepackage{amsmath}
\usepackage{amsfonts}
\usepackage{mathtools}
\usepackage{pgf}
\usepackage{tikz}
\usepackage{listings}
\usepackage{synttree}

\usepackage{makecell}
\usepackage{booktabs}

\usetikzlibrary{positioning}
\usetikzlibrary{arrows,automata}
\usetikzlibrary{shapes}

\DeclarePairedDelimiter\ceil{\lceil}{\rceil}
\DeclarePairedDelimiter\floor{\lfloor}{\rfloor}


\title{\textbf{Ασφάλεια Συστημάτων και Υπηρεσιών} \\ Αναφορά $7^{\eta \varsigma}$ Άσκησης }
\author{ \vspace*{0.5cm} \\ Γιουμερτάκης Απόστολος, 2017030142 \\ }
\date{} 

% \begin{figure}[h!]
%     \centering
%     \includegraphics[width=0.6\linewidth]{graphs/initDog.png}
%     % \caption{\latin{Source Image}}
% \end{figure}
% \vspace*{1cm}

\begin{document}
\gr
\maketitle


\section*{Άσκηση 1}

Κατασκευή πίνακα πολιτικών του τείχους προστασίας
% Για να μετρήσουμε την παραμόρφωση που υπέστει η \latin{convoluted} εικόνα έγινε η χρήση:
% \begin{itemize}
    
%     \item[i)] \latin{Mean square error} που ορίζεται ώς
%         $$MSE(I,\tilde{I}) = \frac{1}{MN}\sum\limits_i^N\sum\limits_j^M (I(i,j) - \tilde{I}(i,j))^2$$

%     \item[ii)] \latin{Peak Signal-to-Noise Ratio} που ορίζεται ως
%         $$PSNR(I,\tilde{I}) = 10 log_{10} \left( \frac{max_I^2}{MSE(I,\tilde{I})} \right)$$
% \end{itemize}

\en
\begin{table}[h!]
\begin{center}
% \begin{tabularx}{1\textwidth} { 
%   | >{\centering\arraybackslash}X 
%   | >{\centering\arraybackslash}X 
%   | >{\centering\arraybackslash}X 
%   | >{\centering\arraybackslash}X 
%   | >{\centering\arraybackslash}X 
%   | >{\centering\arraybackslash}X 
%   | >{\centering\arraybackslash}X 
%   | >{\centering\arraybackslash}X 
%   | >{\centering\arraybackslash}X 
%   | }
%  \begin{tblr}{|c|c|c|c|c|c|c|c|c|}
 \begin{tblr}{| c | c | >{\centering}m{4em} | >{\centering}m{4em} | >{\centering}m{3.5em} | >{\centering}m{3em} | >{\centering}m{3em} | >{\centering}m{3em} | >{\centering}m{4.5em} | >{\centering}m{5.5em} |}
 \hline
 & Action & Source Address & Dest Address & Protocol & Source Port & Dest Port & Flag bit & Check connection & Description \\
 \hline\hline
 1 & Allow & Internet & TUC & TCP & ANY & 80 & ANY & 0 & \makecell{HTTP in\\(server)} \\
 \hline
 2 & Allow & Internet & TUC & TCP & ANY & 443 & ANY & 0 & \makecell{HTTPs in\\(server)}\\
 \hline
 3 & Allow & TUC & Internet & TCP & 80 & ANY & ANY & 0 & \makecell{HTTP out\\(server)}\\
 \hline
 4 & Allow & TUC & Internet & TCP & 443 & ANY & ANY & 0 & \makecell{HTTPs out\\(server)}\\
 \hline
 5 & Allow & Internet & TUC & TCP & 80 & ANY & ANY & 0 & \makecell{HTTP in\\(client)} \\
 \hline
 6 & Allow & Internet & TUC & TCP & 443 & ANY & ANY & 0 & \makecell{HTTPs in\\(client)}\\
 \hline
 7 & Allow & TUC & Internet & TCP & ANY & 80 & ANY & 0 & \makecell{HTTP out\\(client)}\\
 \hline
 8 & Allow & TUC & Internet & TCP & ANY & 443 & ANY & 0 & \makecell{HTTPs out\\(client)}\\
 \hline
 9 & Allow & Internet & TUC & TCP & 22 & ANY & ACK & 1 & \makecell{SSH,\\SFTP in\\(client)}\\
 \hline
 10 & Allow & TUC & Internet & TCP & ANY & 22 & ACK & 1 & \makecell{SSH,\\SFTP out\\(client)}\\
 \hline
%  6 & Allow & Internet & TUC & TCP & ANY & 21 & SYN & 420 & FTP\\
%  \hline
%  7 & Allow & Internet & TUC & TCP & ANY & 995 & SYN & 000 & \makecell{POP3\\(rx enc)}\\
%  \hline
%  8 & Allow & Internet & TUC & TCP & ANY & 993 & SYN & 000 & \makecell{IMAP\\(rx enc)}\\
%  \hline
 11 & Allow & TUC & Internet & \makecell{UDP/\\TCP} & 53 & ANY & ACK & 1 & \makecell{DNS\\server}\\
 \hline
 12 & Allow & Internet & TUC & \makecell{UDP/\\TCP} & ANY & 53 & ANY & 0 & \makecell{DNS\\server}\\
 \hline
 13 & Allow & TUC & Internet & \makecell{UDP/\\TCP} & ANY & 53 & ANY & 0 & \makecell{DNS\\external}\\
 \hline
 14 & Allow & Internet & TUC & \makecell{UDP/\\TCP} & 53 & ANY & ACK & 1 & \makecell{DNS\\external}\\
 \hline
 15 & Reject & Internet & TUC & ICMP & ANY & ANY & ANY & 0 & ICMP in\\
 \hline
 16 & Allow & TUC & Internet & ICMP & ANY & ANY & ANY & 0 & ICMP out\\
 \hline
 17 & Deny & 0.0.0.0 & 0.0.0.0 & TCP & ANY & ANY & ANY & -- & NONE\\
 \hline
 18 & Deny & 0.0.0.0 & 0.0.0.0 & UDP & ANY & ANY & ANY & -- & NONE\\
 \hline
 \end{tblr}
%  \end{tabularx}
  \end{center}
\end{table}
\gr

\newpage
\noindent
\latin{\textbf{Action}}:
\begin{itemize}
\itemsep 0em
    \item \latin{Allow}: Επιτρέπει την πρόσβαση στα συγκεκριμένα \latin{ports}.
    \item \latin{Allow}: Απότρέπει την πρόσβαση στα συγκεκριμένα \latin{ports}, χωρίς καμία απάντηση στον χρήστη (\latin{client}).
    \item \latin{Reject}: Απότρέπει την πρόσβαση στα συγκεκριμένα \latin{ports}, στέλνοντας την απάντηση "\latin{Destination is unreachable}" στον χρήστη (\latin{client}).
\end{itemize}

\noindent
\textbf{Κανόνας 1,2:}\\
Επιτρέπει την αποστολή πακέτων από εξωτερικό υπολογιστή πρός το Πολυτεχνείο Κρήτης για πρωτόκολλα \latin{HTTP, HTTPs} για πρόσβαση σε ιστοσελίδες όπως το \latin{tuc.gr}, με όλα τα \latin{flags} ενεργοποιημένα. \\

\noindent
\textbf{Κανόνας 3,4:}\\
Επιτρέπει την αποστολή πακέτων (απαντήσεων) από \latin{servers} του Πολυτεχνείου Κρήτης προς εξωτερικό υπολογιστή για πρωτόκολλα \latin{HTTP, HTTPs} για πρόσβαση σε ιστοσελίδες όπως το \latin{tuc.gr}, με όλα τα \latin{flags} ενεργοποιημένα. \\

\noindent
\textbf{Κανόνας 5,6:}\\
Επιτρέπει την αποστολή πακέτων από υπολογιστές του Πολυτεχνείου Κρήτης προς εξωτερικούς \latin{servers} για πρωτόκολλα \latin{HTTP, HTTPs} για πρόσβαση σε ιστοσελίδες όπως το \latin{google.com}, με όλα τα \latin{flags} ενεργοποιημένα. \\

\noindent
\textbf{Κανόνας 7,8:}\\
Επιτρέπει την αποστολή πακέτων (απαντήσεων) από εξωτερικούς \latin{servers} προς υπολογιστές του Πολυτεχνείου Κρήτης για πρωτόκολλα \latin{HTTP, HTTPs} για πρόσβαση σε ιστοσελίδες όπως το \latin{google.com}, με όλα τα \latin{flags} ενεργοποιημένα. \\

\noindent
\textbf{Κανόνας 9:}\\
Επιτρέπει την αποστολή πακέτων (απαντήσεων) από εξωτερικούς \latin{servers} προς υπολογιστές του Πολυτεχνείου Κρήτης για πρόσβαση με \latin{secure shell} (πρωτόκολλα \latin{SSH, SFTP} και οποιαδήποτε υπηρεσία υλοποιείται πάνω από \latin{ssh}). \\

\noindent
\textbf{Κανόνας 10:}\\
Επιτρέπει την αποστολή πακέτων από υπολογιστές του Πολυτεχνείου Κρήτης προς εξωτερικούς \latin{servers} για πρόσβαση με \latin{secure shell} (πρωτόκολλα \latin{SSH, SFTP} και οποιαδήποτε υπηρεσία υλοποιείται πάνω από \latin{ssh}). \\

\noindent
\textbf{Κανόνας 11:}\\
Επιτρέπει την αποστολή πακέτων (απαντήσεων) από τον \latin{server} του Πολυτεχνείου Κρήτης προς εξωτερικούς υπολογιστές για αναζήτηση πληροφοριών \latin{DNS}, με τα πρωτόκολλα \latin{UDP, TCP} αντίστοιχα (τo πρωτόκολλo \latin{UDP} πιο διαδεδομένο από το \latin{TCP} για εφαρμογές \latin{DNS}). Το \latin{flag: ACK} έχει νόημα μόνο όταν το πρωτόκολλο είναι \latin{TCP}.\\

\noindent
\textbf{Κανόνας 12:}\\
Επιτρέπει την λήψη πακέτων (\latin{queries}) από εξωτερικούς υπολογιστές προς τον \latin{server} του Πολυτεχνείου Κρήτης για αναζήτηση πληροφοριών \latin{DNS}, με τα πρωτόκολλα \latin{UDP, TCP} αντίστοιχα (τo πρωτόκολλo \latin{UDP} πιο διαδεδομένο από το \latin{TCP} για εφαρμογές \latin{DNS}).\\

\noindent
\textbf{Κανόνας 13:}\\
Επιτρέπει την αποστολή πακέτων (\latin{queries}) από υπολογιστές του Πολυτεχνείου Κρήτης προς εξωτερικούς \latin{servers} για αναζήτηση πληροφοριών \latin{DNS}, με τα πρωτόκολλα \latin{UDP, TCP} αντίστοιχα.\\

\noindent
\textbf{Κανόνας 14:}\\
Επιτρέπει την λήψη πακέτων (απαντήσεων) από εξωτερικούς \latin{servers} προς υπολογιστές του Πολυτεχνείου Κρήτης για αναζήτηση πληροφοριών \latin{DNS}, με τα πρωτόκολλα \latin{UDP, TCP} αντίστοιχα. Το \latin{flag: ACK} έχει νόημα μόνο όταν το πρωτόκολλο είναι \latin{TCP}.\\

\noindent
\textbf{Κανόνας 15:}\\
\textbf{Απότρέπει} την λήψη πακέτων από εξωτερικούς υπολογιστές προς \latin{webservers} του Πολυτεχνείου Κρήτης για έλεγχο και διαχείρηση των συστημάτων, με το πρωτόκολλο \latin{ICMP}. \\

\noindent
\textbf{Κανόνας 16:}\\
Επιτρέπει την αποστολή πακέτων από υπολογιστές του Πολυτεχνείου Κρήτης προς εξωτερικούς \latin{servers} για έλεγχο και διαχείρηση των συστημάτων, με το πρωτόκολλο \latin{ICMP}. \\

\noindent
\textbf{Κανόνας 17,18:}\\
Μπλοκάρεται όλη η κίνηση για οποιαδήποτε άλλη υπηρεσία "τρέχει" σε \latin{ports}.

\section*{Άσκηση 2}
Ερώτηση:\\
\textit{Σε περίπτωση που υλοποιούσατε το συγκεκριμένο τείχος προστασίας σε ένα \latin{Linux PC},
ποιος θα ήταν ο ελάχιστος αριθμός καρτών \latin{Ethernet} που θα έπρεπε να είχε το \latin{PC}.
Δικαιολογήστε σε μία γραμμή την απάντησή σας.}\\
\newline
Απάντηση:\\
Χρειάζονται τουλάχιστον δύο(2) κάρτες για την λειτουργία του τείχους προστασίας, μια με στατική τοπική διεύθυνση \latin{IP (147.27.x.x)} και μια με εξωτερική διεύθυνση \latin{IP}, ώστε να μπορεί να δει τις διευθύνσεις κάθε εσωτερικού αλλά και εξωτερικού υπολογιστή.\\
\newline
\newline
\newline
Tα παραπάνω επιβεβαιώθηκαν στο μέγιστο δυνατό χρησιμοποιώντας το πρόγραμμα καταγραφής πακέτων, \latin{Wireshark}.

\end{document}